\section{Favorite-Longshot Bias}
\label{appendix:favorite_longshot}

The Favorite-Longshot Bias is a phenomenon observed in gambling markets where the expected returns on bets placed on longshots (i.e., outcomes with low probabilities of occurrence) are significantly lower than those of favorites (i.e., outcomes with high probabilities of occurrence). Despite the lower returns, bettors disproportionately favor longshots.

This bias has been documented in various betting markets, including horse racing, sports betting, and lottery games. The bias suggests that bettors tend to overestimate the chances of unlikely events and underestimate the chances of likely events.

While there is no consensus explanation for the favorite-longshot bias \cite{ottaviani_2008_chapter}, several theories have been proposed. Some suggest that it may be due to misperceptions of probabilities, while others attribute it to the utility of gambling (i.e., the thrill of potentially winning a large amount from a small stake).

In the context of a carnival or similar event, understanding this bias can be useful for game organizers. By offering large potential prizes for unlikely outcomes, they can attract more participants and increase overall profits, even if the games are statistically rigged in favor of the organizer. This is because the possibility of winning a large prize can make the risk of losing seem worthwhile to many participants.
