\subsubsection{Risk 5: Leftover Prizes}

After hosting the carnival, there is a risk that there will be a surplus of prizes that were not won. This could lead to logistical challenges due to an accumulation of unused inventory. Given that this is a one-time event, any leftover inventory would have no further value. To mitigate this risk, we have devised a comprehensive liquidation strategy.

Firstly, we plan to participate in our school's swap meet, a platform where vendors can sell their items. Here, we intend to sell our leftover prizes. To make this process more appealing, we will host a raffle for the most popular prizes.

To maximize participation in the raffle, we have taken into account findings from a study by Jeffrey Carpenter and Peter Hans Matthews \cite{carpenter_2017_using}. The study found that perceived fairness could increase raffle participation. Therefore, we will ensure transparency about how the raffle winner will be selected and where the profits from the raffle will go. We will price the raffle tickets at one dollar each to facilitate easy purchase.

Secondly, we are considering selling our leftover prizes on eBay. However, due to eBay's transaction fees, we will prioritize sales to CCA students. For instance, an item that sells for \$10 will have around \$3.50 deducted for shipping and another 13.25\% deducted for eBay's transaction fees. This fee structure, which can be found in Appendix \ref{appendix:ebay_fees}, will be transparently communicated to potential buyers.

Through these measures, we aim to ensure the efficient liquidation of leftover prizes, thereby mitigating potential financial losses and storage issues.
