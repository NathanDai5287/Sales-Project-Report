\subsubsection{Risk 3: Disincentivized Players}

A significant risk we identified pertains to the incentivization of play. The carnival games we selected from the AP Statistics class are designed to statistically favor the game organizer. While individual play involves an element of chance, the Law of Large Numbers dictates that the probability of the game operator incurring a loss decreases as the number of players increases [\ref{appendix:large_numbers}].

However, a considerable portion of our carnival attendees were friends who were aware that the games were skewed in favor of the operators. We were concerned that this knowledge might deter participation. The potential impact of this risk was substantial, as reduced player engagement could lead to lower ticket sales and diminished overall enjoyment of the carnival.

To mitigate this risk, we conducted research into the economics of various gambling institutions and discovered the favorite-longshot bias. This well-documented phenomenon reveals that people often overestimate the impact and likelihood of extremely low-probability events [\ref{appendix:favorite_longshot}]. We concluded that individuals are drawn to risk when it is accompanied by the potential for substantial returns. Thus, in a gambling transaction, both the bettor and the bet acceptor find their actions worthwhile. Our carnival benefits by generating overall profits, while participants derive enjoyment from the prospect of winning a large prize.

Furthermore, it is improbable that an individual would discern that our carnival games are designed to favor the operators purely through playing the game. Each player can attribute their losses to random chance, while we, as the organizers, have a broader perspective. This understanding informed our risk mitigation strategy, ensuring that our carnival remains both profitable and enjoyable for all participants.
