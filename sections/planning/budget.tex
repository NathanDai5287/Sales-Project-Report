\subsection{Proposed Project Budget}

Our initial funding strategy involves leveraging the crowdfunding platform, GoFundMe. We intend to compile a list of email addresses from the club's parent community and initiate a campaign requesting their generous contributions. Additionally, we aim to engage with DECA alumni, particularly past officers, inviting them to financially support our club's endeavors. The funds raised will be allocated towards the procurement of prizes for the carnival.

Our procurement strategy is to source prizes from Temu, a cost-effective alternative to Amazon, albeit with a longer delivery timeframe. To mitigate this, we will plan our carnival and place prize orders well in advance. The prizes we select will span a range of values, from nominal to moderate, ensuring that no single game carries the risk of a significant financial loss for the carnival game operators. Appendix \ref{appendix:temu} provides a list of prizes we intend to purchase and their respective prices, totaling \$111.43 before tax.

In the event that the contributions from GoFundMe, parents, and alumni do not suffice, we propose to establish partnerships with local food vendors to set up food truck fundraisers. We would invite these vendors, potentially including Kona Ice and Wetzel Pretzel, to operate food trucks on campus. A portion of the profits from these operations would be allocated to our club.

Furthermore, we plan to collaborate with local restaurants such as Happy Lemon and Chipotle to host fundraising events. During these events, a percentage of the profits from orders placed by students in support of our club would be directed towards our financial goals. This strategy not only aids our fundraising efforts but also fosters community engagement and support for our club.
